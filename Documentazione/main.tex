\documentclass[12pt]{report}
\usepackage[utf8]{inputenc}
\usepackage{amsmath}
\usepackage{amsfonts}
\usepackage{venndiagram}
\usepackage{hyperref}
\usepackage{multirow}

\usepackage{geometry}
\geometry{margin=1.1in}

\title{Fondamenti di Intelligenza Artificiale}
\author{Alessandro Satta, Roberto Piscopo}
\date{2023}

\begin{document}
\maketitle
\tableofcontents

\chapter{Introduzione}
\section{Contesto}    
Il progetto nasce dal problema che molte persone hanno,ovvero avere in dispensa/frigo degli ingredienti dimenticati oppure accumulati e non sapere cosa cucinare. 

\section{Idea}
Il nostro progetto ha lo scopo di creare un agente intelligente capace di consigliare una o più ricette in base a quello che l’utente ha a disposizione  

\section{Problemi}
Analizzando la nostra idea è sorto un grosso problema: l’utente forse ha tutti gli ingredienti di quella ricetta consigliata ma non ha forse 1 ora per preparala. Quindi ci siamo accorti che dovevamo includere un altro dettaglio ovvero il tempo di preparazione. 

La prima cosa da fare quando si progetta un agente intelligente è la definizione delle Specifiche Peas 

\chapter{PEAS}
\section{Specifiche}
La prima cosa da fare quando si progetta un agente intelligente è la definizione delle Specifiche Peas 

 Performance: Migliore ricetta in base agli ingredienti forniti, privilegiando quella con il numero di likes maggiore e tempo di preparazione minore. 

Ambiente:  

L’ambiente nel quale il nostro agente opera è composto da tutte le ricette con i relativi : id, titolo, ingredienti, likes, tempo di preparazione e calorie. 

L’ambiente risulta: 

Completamente osservabile: si ha accesso a tutte le informazioni di ogni ricetta 

deterministico: l'agente sa esattamente cosa aspettarsi dall'ambiente in base alle informazioni fornite e agli input dell'utente. 

Oppure? 

Stocastico: L’agente compirà azioni specifiche che possono portare a risultati e stati differenti, e quindi lo sato successivo dell’ambiente risulta essere incerto 

L'ambiente deterministico significa che l'agente sa esattamente cosa aspettarsi in base alle informazioni fornite e agli input dell'utente. In altre parole, l'agente sa che dati gli stessi input, otterrà sempre lo stesso output, l'agente sa che se gli vengono forniti gli stessi ingredienti, troverà sempre la stessa ricetta migliore nell'ambiente statico 

episodico: l'agente si trova in un singolo episodio in cui riceve gli ingredienti in input e restituisce la migliore ricetta come output, senza alcuna interazione ulteriore con il mondo esterno. 

Statico: L’insieme delle ricette e le relative informazioni non cambiano nel corso del tempo 

Oppure Semi-Dinamico: L’insieme delle ricette non cambia, ma il comportamento dell’agente sull’ambiente cambia 

Attuatori: L'algoritmo utilizzato per generare le possibili soluzioni e selezionare la migliore. 

Discreto: L’agente ha a disposizione un numero limitato di azioni 

Sensori: Gli input dell'utente, ovvero la lista degli ingredienti. 

Singolo Agente: è presente solo un agente che opera nell’ambiente 

\chapter{Dataset}
\section{Creazione Dataset}
Non esistendo dei dataset di ricette in italiano, abbiamo dovuto crearlo noi. Per farlo abbiamo utilizzato dei tool online. Il nostro dataset ha circa 300 ricette ed è un file CSV di questo tipo: 
\begin{center}
\begin{tabular}{|c|c|c|c|c|c|} 
 \hline
 ID & Titolo & Ingredienti & likes & Prep time & Calorie \\ [0.5ex] 
 \hline
 335 & Gnocchi di zucca  & Zucca, farina, uova, noce moscata, parmigiano  & 200 & 45 min & 600 \\
 \hline
 160 & Falooda & noodles,latte,rosewater,latte di cocco,frutta  & 150 & 20 min & 400 \\
 \hline
175 & Tortilla spagnola & patate,cipolla,uova  & 150 & 20 min & 350 \\
 \hline
342 & Pasta e Fagioli & pasta,fagioli,pomodoro,cipolla,aglio,olio  & 170 & 15 min & 300 \\
 \hline
\multicolumn{6}{|c|}{ecc...} \\
 \hline
\end{tabular}
\end{center}


\section{Data Preparation}

\section{Data Cleaning}
 Essendo il dataset creato da noi, non ha necessitato di una rimozione di ricette ridondanti o di dati mancanti. 

\section{Feature Selection}
Come già detto , il dataset delle ricette è stato creato e quindi di seguito vengono spiegati i motivi di ogni caratteristica: 
\begin{itemize}
	\item Id: utile per identificare una specifica ricetta 
	\item Titolo: è utile per mostrare all’utente la ricetta consigliata 
	\item Ingredienti: è la feature dominante in quanto
	\item Like: 
	\item Prep time:
	\item Calorie:
 \end{itemize}

\end{document}
